\lstdefinelanguage{bash}{
  keywords={if, else, for, while, do, done, exit, then, fi, mkdir, echo, srun},
  keywordstyle=\color{blue}\bfseries,
  ndkeywords={sudo, cp, rm, touch, python3},
  ndkeywordstyle=\color{magenta}\bfseries,
  identifierstyle=\color{black},
  sensitive=false,
  comment=[l]{\#},
  morecomment=[s]{\#\#}{\#\#},
  commentstyle=\color{gray}\ttfamily,
  stringstyle=\color{darkgreen}\ttfamily,
  morestring=[b]",
  morestring=[b]'
}

\section*{Appendix - Introductory exercise}
\label{app:pingpong}
The following code was used for the ping pong task:
\lstinputlisting[language=c]{input/code/00/pingPong.c}
For the bonus task, the following code was used:
\lstinputlisting[language=c]{input/code/00/pingPongBonus.c}

\label{app:mm}
The matrix multiplication used the following code: 
\lstinputlisting[language=c]{../../01_lab0/src/MM-product.c}

\section*{Appendix - Poisson solver}
\label{app:poisson}
The parallel Poisson solver used the following code:\\
\underline{Note:} Sbatch scripts used for the exercises will be included after the Poisson-solver code.\\
\lstinputlisting[language=c]{../../02_lab1/src/MPI_Poisson.c}
Script used for \autoref{subsec:scaling}:
\lstinputlisting[language=bash]{../../02_lab1/scripts/scaling_123.sh}